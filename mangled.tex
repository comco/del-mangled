\documentclass[a4paper]{article}
\usepackage[utf8]{inputenc}
\usepackage[english]{babel}
\usepackage{amsmath,amsthm,amssymb}
\usepackage{geometry}

\title{Mangled Children \\ \large Quantum exploration of a classical epistemic
 puzzle}
\author{Krasimir Georgiev}

\begin{document}
\maketitle
\section*{Introduction} The Muddy children puzzle is a well-known epistemic
puzzle that illustrates some non-trivial and at first sight even paradoxical
features of the dynamics of knowledge. It can however be easily modeled as a
relational epistemic model and analyzed completely using basic dynamic epistemic
logic. Quantum physics is normally perceived as a notoriously unintuitive
discipline. It lacks connections with people's real world intuitions and
predicts bizarre, paradoxical behaviours. In this report we explore quantum
analogues of the Muddy children puzzle. These variants sometimes exhibit bizarre
behaviours, and we use epistemic techniques and concepts to shed some light on
the reasons behind them.

\section*{Muddy children} The classical Muddy children puzzle goes as follows:
several clever (logically omniscient) children are playing with mud; here comes
the father and tells them (publicly announces): some of you have mud on their
foreheads. Then he proceeds to ask the same question: do you know if you have
mud on your forehead; each time the children all answer (publicly announce) at
the same time truthfully. Each child can see everyone else's foreheads, except
for its own. During the first several rounds, everyone answers negatively. But
then, at some round, exactly the muddy children answer positively. This is
seemingly paradoxical: by listening to the same answers of the same question
over and over again, the children come to learn more.

For $n$ children, we may model the puzzle by an epistemic model where the worlds
are all the strings over $\{0, 1\}$ of length $n$, where $1$ at position $i$
denotes that the $i$-th child has mud on its forehead (and $0$ that it's clean).
The indistinguishability relation $R_i$ for the $i$-th child contains all the
pairs of strings that possibly differ only at the $i$-th position.
Suppose that $u$ of the children are actually muddy. The public announcement of
the father effectively deletes the world $0^n$. At round $r = 1, 2, \dots, u-1$,
the public announcements of the children delete all the worlds having $r$ ones.
At round $r = u$, since all worlds with a lower number of ones are deleted, all
the muddy children can distinguish the real world, so they give a positive
answer.

\section*{Quantum reading}
A postulate of quantum physics is that systems can be modelled as unit vectors
(or equivalently, one dimensional subspaces) in some Hilbert space.
In the quantum world, observations are both limited and limiting. Only the
``testable'' properties (corresponding to closed subspaces of the Hilbert space) 
can possibly be observed, and observing them leads to a real, ontic ``collapse''
of the quantum system as a projection into the testable property subspace. On
top of that the result of the observation is probabilistic, making most of the
implicit information in the quantum state inaccessible by observations.

\section*{Qubit foreheads}

\section*{Shared mud}

\section*{Spin mud}

\end{document}
