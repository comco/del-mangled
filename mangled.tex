\documentclass[a4paper]{article}
\usepackage[utf8]{inputenc}
\usepackage[english]{babel}
\usepackage{amsmath,amsthm,amssymb}
\usepackage{geometry}
\usepackage{braket}

\newcommand{\HH}{\mathcal{H}}

\title{Mangled Children \\ \large Quantum exploration of a classical epistemic
 puzzle}
\author{Krasimir Georgiev}

\begin{document}
\maketitle
\section*{Introduction} The Muddy children puzzle is a well-known epistemic
puzzle that illustrates some non-trivial and at first sight even paradoxical
features of the dynamics of knowledge. It can however be easily modeled as a
relational epistemic model and analyzed completely using basic dynamic epistemic
logic. Quantum physics is normally perceived as a notoriously unintuitive
discipline. It lacks connections with people's real world intuitions and
predicts bizarre, paradoxical behaviours. In this report we explore quantum
analogues of the Muddy children puzzle. These variants sometimes exhibit bizarre
behaviours, and we use epistemic techniques and concepts to shed some light on
the reasons behind them.

\section*{Muddy children} The classical Muddy children puzzle goes as follows:
several clever (logically omniscient) children are playing with mud; here comes
the father and tells them (publicly announces): some of you have mud on their
foreheads. Then he proceeds to ask the same question: do you know if you have
mud on your forehead; each time the children all answer (publicly announce) at
the same time truthfully. Each child can see everyone else's foreheads, except
for its own. During the first several rounds, everyone answers negatively. But
then, at some round, exactly the muddy children answer positively. This is
seemingly paradoxical: by listening to the same answers of the same question
over and over again, the children come to learn more.

For $n$ children, we may model the puzzle by an epistemic model where the worlds
are all the strings over $\{0, 1\}$ of length $n$, where $1$ at position $i$
denotes that the $i$-th child has mud on its forehead (and $0$ that it's clean).
The indistinguishability relation $R_i$ for the $i$-th child contains all the
pairs of strings that possibly differ only at the $i$-th position.
Suppose that $u$ of the children are actually muddy. The public announcement of
the father effectively deletes the world $0^n$. At round $r = 1, 2, \dots, u-1$,
the public announcements of the children delete all the worlds having $r$ ones.
At round $r = u$, since all worlds with a lower number of ones are deleted, all
the muddy children can distinguish the real world, so they give a positive
answer.

\section*{Quantum reading}
A postulate of quantum physics is that quantum states can be modelled as unit 
vectors
$s \in \mathcal{H}, |s| = 1$ in some Hilbert space. This is equivalent (up to a
sign, which is insignificant with respect to measurements) with the one
dimensional subspaces interpretation of quantum states, and it more easy to work
with numerically.

In the quantum world, observations are both limited and limiting. Only the
``testable'' properties (corresponding to closed subspaces of the Hilbert space) 
can possibly be observed, and observing them leads to a real, ontic ``collapse''
of the quantum system as a projection into the testable property subspace. On
top of that the result of the observation is probabilistic, making most of the
implicit information in the quantum state inaccessible for observers.

These basic quantum principles force us to reconsider the basic ingredients of
the Muddy children puzzle if we hope to achieve a quantum version of it. Since
the detection of mud amounts to an observation of the quantum system, which
collapses it, there may be no definite presence or absence of mud on the
children's foreheads at all. In general, the forehead might exist in a
superposition of the presence and absence of mud, that is additionally entangled
with the foreheads of the other children. Since the observation updates the
initial quantum state irreversibly, it no longer makes much sense to talk about
the absence or presence of mud on any child's forehead in general. A possible
adaptation is: to say ``if you observe your forehead, it is possible that you
will see some mud''. Another variation might be: ``if you observe your forehead,
you will surely see some mud'', or even ``you will observe your forehead being
muddy with probability $P \in [0, 1]$''. Let us see what these mean in terms of
the states of the system. Let $s \in \HH$ be unit vector corresponding to the 
real state of the system and let $m \in \HH$ be the unit vector corresponding to
subspace of the forehead observation. The first interpretation corresponds to
the non-orthogonality of $s$ and $m$: $s \not\perp m$. The second corresponds to
the perpendicularity of $s$ and $m$, and the third corresponds to the following:
in an orthonormal basis for $\HH$ containing $m$, the state $s$ will have a
coefficient (amplitude) $\alpha \in \mathbb{C}$ along $m$ for which the
probability is the square of the amplitude: $P = |\alpha|^2$. In this version,
we also need to assume that there are infinitely many copies of the initial
quantum system, and after performing infinitely many observations, the
probability in the limit for a muddy forehead will reach $P$.

Another issue that pops up is the simultaneous observations. Since an
observation has an ontic effect, it is in general impossible for all observers
to observe the system at the same time. Only in specific cases (as in the next
section) will it be possible to combine two observable properties and to get an
observable property. So in general, the interpretation of disjunction might be
problematic, because it might lead to a non-observable property. We may need to
introduce a total order between the observations, so that the system first
collapses to the subspace of the first observation first, then to the subspace
of the next, and so on. Note that this holds only for the quantum observation
part of the action of a child; it is still possible for them, after having made
all observations, to answer to the question posed by the father classically
simultaneously (at least up to the epistemic characteristics of their answers).
For convenience, we will number the children by $1, 2, \dots, n$ and we will
assume that they make their observations in this order.

TODO
Also the question ``do you know if you are dirty'' needs an adaptation. It is 
``after doing your quantum observation, do you know if you did an observation of
your forehead, you will be dirty''.

\section*{Qubit mud}
Let us consider a situation in which there are just $n = 2$ children, Alice and
Bob and there is a qubit for the mud on each one's forehead. Alice's forehead
has two possible observable states: $\ket{a}$, in which case her forehead is
clean, and $\ket{A}$, in which case she is dirty. Similarly for Bob the states
are $\ket{b}$ and $\ket{B}$. A general state of this system is a unit vector in
a four-dimensional Hilbert space $\HH$ having the form $s = \delta \ket{ab} +
\beta \ket{aB} + \alpha \ket{Ab} + \gamma \ket{AB}$, where $\alpha, \beta,
\gamma, \delta \in \mathbb{C}$ and $|\alpha|^2 + |\beta|^2 + |\gamma|^2 +
|\delta|^2 = 1$.  Assume that this is common knowledge, so our initial epistemic
model contains all these states as possible worlds.  Also, since observation has
an ontic effect, the kids don't really look at each other's foreheads, so the
indistinguishability relation $R_i$ is the full relation for both of them.

Then comes the quantum omniscient father (he is cool because he can see the
complex coefficients along the basis vectors \emph{without} disturbing the
system), and tells them: some of you has mud on its forehead. In this particular
case, there is a yet another interesting quantum interpretation of this statement:
that the coefficient $\delta$ is zero. Observe that this is a different
interpretation than any of the interpretations discussed in the previous
section. Here, since the property of absence of mud on both foreheads actually
corresponds to the basis vector $\ket{ab}$, it is testable, and we give the
rough interpretation as: ``well kids, in this particular situation, as you can
see, you may in principle observe both foreheads at the same time, and if you
did that, you'll surely not observe two clean ones''. This public announcement
effectively deletes the states having $\delta \neq 0$, leaving us with an
epistemic model containing states that are the unit vectors of the form 
$s = \beta \ket{aB} + \alpha \ket{Ab} + \gamma \ket{AB}$.

Now the father asks the children if they know if they are dirty (after a quantum
observation) and Alice observes Bob's forehead. By doing so, she collapses the
system to a state that is consistent to the answer she gets. In detail:
\begin{itemize}
    \item She observes that Bob is muddy, which might happen with probability 
        $|\beta|^2 + |\gamma|^2$. In this case, the system  collapses to the state
        $s' = \frac{\beta \ket{aB} + \gamma \ket{AB}}{|\beta|^2 + |\gamma|^2}$.
        She confuses all the states of this form, and she answers ``no'' to the
        question of the father. TODO: BOB?
    \item She observes that Bob isn't muddy, which might happen with probability
        $|\alpha|^2$. In this case, the system collapses to the state $s'' =
        \frac{\alpha \ket{aB}}{|\alpha|^2}$, which is a unit complex multiple of
        the basis vector $\ket{aB}$. She of course confuses all the states of
        this form, but she knows that the only possible outcome of measuring her
        dirtiness at any state of the form $s''$ must be $\ket{a}$, so she now
        knows that she is dirty and answers ``yes'' to the question of the
        father. TODO: BOB?
\end{itemize}
\section*{Shared mud}

\section*{Spin mud}

\end{document}
